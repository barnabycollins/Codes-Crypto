% !TEX TS-program = pdflatex
% !TEX encoding = UTF-8 Unicode

% This is a simple template for a LaTeX document using the "article" class.
% See "book", "report", "letter" for other types of document.

\documentclass[11pt]{article} % use larger type; default would be 10pt

\usepackage[utf8]{inputenc} % set input encoding (not needed with XeLaTeX)

%%% Examples of Article customizations
% These packages are optional, depending whether you want the features they provide.
% See the LaTeX Companion or other references for full information.

%%% PAGE DIMENSIONS
\usepackage[a4paper,left=2cm,right=2cm,top=2cm,bottom=2cm]{geometry}
%\usepackage{geometry} % to change the page dimensions
% \geometry{a4paper} % or letterpaper (US) or a5paper or....
% \geometry{margin=0in} % for example, change the margins to 2 inches all round
% \geometry{landscape} % set up the page for landscape
%   read geometry.pdf for detailed page layout information

\usepackage{graphicx} % support the \includegraphics command and options

\usepackage[parfill]{parskip} % Activate to begin paragraphs with an empty line rather than an indent

%%% PACKAGES
\usepackage{booktabs} % for much better looking tables
\usepackage{array} % for better arrays (eg matrices) in maths
\usepackage{paralist} % very flexible & customisable lists (eg. enumerate/itemize, etc.)
\usepackage{verbatim} % adds environment for commenting out blocks of text & for better verbatim
\usepackage{subfig} % make it possible to include more than one captioned figure/table in a single float
\usepackage{amsmath}
\usepackage{amssymb}
\usepackage{logicproof}
\usepackage{tikz}
\usetikzlibrary{arrows,petri,topaths}
\usepackage{float}
\usepackage{graphicx}
\usepackage[T1]{fontenc}
% These packages are all incorporated in the memoir class to one degree or another...

%%% HEADERS & FOOTERS
\usepackage{fancyhdr} % This should be set AFTER setting up the page geometry
\pagestyle{fancy} % options: empty , plain , fancy
\renewcommand{\headrulewidth}{0pt} % customise the layout...
\lhead{}\chead{}\rhead{}
\lfoot{}\cfoot{\sffamily\thepage\normalfont}\rfoot{zrlr73}

%%% SECTION TITLE APPEARANCE
\usepackage{sectsty}
\allsectionsfont{\sffamily\mdseries\upshape} % (See the fntguide.pdf for font help)
% (This matches ConTeXt defaults)

%%% ToC (table of contents) APPEARANCE
\usepackage[nottoc,notlof,notlot]{tocbibind} % Put the bibliography in the ToC
\usepackage[titles,subfigure]{tocloft} % Alter the style of the Table of Contents
\renewcommand{\cftsecfont}{\rmfamily\mdseries\upshape}
\renewcommand{\cftsecpagefont}{\rmfamily\mdseries\upshape} % No bold!
\newcommand{\qedsymbol}{\rightline{$\blacksquare$}}
\renewcommand{\familydefault}{\sfdefault}
\renewcommand{\thesection}{Idea \arabic{section}:}

\usepackage[style=numeric-comp]{biblatex}

%%% END Article customizations

%%% The "real" document content comes below...

\title{\vspace{-1.6cm}Compressing a LaTeX file}
\author{zrlr73}
\date{} % Activate to display a given date or no date (if empty),
         % otherwise the current date is printed 

\begin{document}
\maketitle



\section{Huffman Coding}

My first instinct was to get a 'baseline' to compare other results against. I chose to achieve this using the \verb|dahuffman| library. I simply ran this library on the input file and pickled the tuple \verb|(table, encoded)| where \verb|table| was the translation table generated by \verb|dahuffman| and \verb|encoded| was the encoded output. On the decoder, this tuple was unpickled and decoded using the same library.

This gained me a compression ratio of around 1.6 on both test documents.


\section{Lempel-Ziv-Welch}

I next decided to try and implement a more advanced encoding scheme, and chose LZW because I like the fact that it generates its decoding dictionary as it goes rather than having to store it with the output. I also chose it over LZ78 because its compressed output is simpler and cleaner thanks to the fact that it pre-fills its dictionary.

While the implementation of the LZW encoder went fairly smoothly, I did struggle with one thing: my decoder kept breaking, because it was looking for elements that did not exist in the dictionary yet. I struggled with this for a while, until I looked up the issue and found that it was in fact a quirk of LZW: if the encoder adds something to the dictionary and then immediately uses it in the next step, that item will not exist yet in the decoder's dictionary when it reaches that point. However, because this is the only time at which an error like this is encountered, it is possible to infer the situation and solve it simply by appending the first character of the last decoded value to the end of that same last decoded value.

Once this was resolved, LZW yielded a compression ratio of 1.8-1.9 on both test documents.


\section{Collapsing repeated characters}

One of my test files contained some series of repeated characters. Indeed, a series of space characters was in the top 3 most common substrings at all reasonable lengths. I decided to write a function to detect these repeated characters, and then replace each run with a single instance of the character being repeated followed by a Unicode character signifying the number of repeats being represented. This character was chosen simply as corresponding character at space \verb|255 + num_repeats| in Unicode. For instance, the string \verb|`xxxxxx'| would be replaced by \verb|`xą'| (where \verb|ą| is the character with code $261 = 255 + 6$).

Preprocessing the string with this method before LZW yielded consistent compression ratio improvements of 0.01-0.05.


\section{Substring replacement: common substring search} \label{commonSubstringSearch}

My next idea was to find the most common substrings in the file and replace them with Unicode characters, with a translation table being sent alongside the original document content. I started writing some algorithms that generated counts of every substring in the file and then filtered them such that only the most useful ones remained.

I used a value of \verb|(length_of_string - 1) * number_of_appearances_in_input| to score candidates, with the subtraction of $1$ accounting for the character replacing the string. While I found this formula to be effective, I noticed that my filtering algorithms were failing to remove a lot of similar substrings, which would be difficult to robustly filter out to retrieve the most efficient of the bunch. For example, it might find "\verb|\begin{ali|" and "\verb|gin{align}|" as well as "\verb|\begin{align}|", the actual targeted string. For this reason, I abandoned this method and used what I'd learned to try a similar, but more targeted, approach.


\section{Substring replacement: regex search}

Following idea 5, I decided to take a more proactive approach in defining what repeated substrings I was after. I decided to use regular expressions to define likely repeating substrings and initially chose the following:

\begin{itemize}
	\item \verb|(\\[a-zA-Z]+)|
	\item \verb|(\\[a-zA-Z]+(?:\{[a-zA-Z ]*\})+)|
	\item \verb|[ \n]([a-zA-Z]+)[ \.,:;!]|
\end{itemize}

These encode a simple \rmfamily\LaTeX\normalfont\, command, a command with any number of parameters contained within \verb|{}| brackets and a single word respectively. Through some experimentation I have adjusted them to the following, finding that (unsurprisingly) increasing the number of matches and targeting longer words improves compression ratios:

\begin{itemize}
	\item \verb|(\\[a-zA-Z]+)|
	\item \verb|(\\[a-zA-Z]+(?:[\{\[][a-zA-Z 0-9]*[\}\]])+)|
	\item \verb|[ \n\(\{]([a-zA-Z]{3,})[ \.,:;!\)\}]|
\end{itemize}

I find all the matches of each regex in the input, produce counts, score each matching string and then sort each regex in descending order of score. I then produce an allocation of a fixed number of slots in a dictionary (initially 512, though this is subject to change) to maximise the total score of the items in the dictionary, and then replace every instance of each matched string with an allocated Unicode character. For this I have to adjust the lower bound of idea 3, essentially producing the following Unicode character allocation:

\begin{itemize}
	\item 0-255: Standard document characters
	\item 256-767: Characters representing replaced common strings
	\item 768+: Characters used to identify the counts of repeated characters
\end{itemize}

As stated before, I am assuming that the document only uses characters 0-255, but this scheme could theoretically also be used for extended character sets that include, for example, Nordic letters - the ranges would change, but the same techniques could still be applied.

The dictionary is then stored alongside the compressed output, making decoding trivial. The function has been written to work with whatever space and regexes are given in order to be easy to tinker with, so it should be easy to adjust and optimise.

Applying this method in between idea 3 and LZW during the compression afforded me compression ratios of 2.1-2.3.









\end{document}
